\documentclass[11pt,a4paper]{article}

\usepackage{titling}
\usepackage[hidelinks]{hyperref}
\usepackage{graphicx}
\usepackage{grffile}
\usepackage{float}
\usepackage{geometry}
\usepackage{listings}

\newcommand{\subtitle}[1]{
  \posttitle{
    \par\end{center}
    \begin{center}\large#1\end{center}
    \vskip0.5em}
}

\begin{document}


\title{Hyper Perform\\ User Manual}
\subtitle{ Repository: \url{https://github.com/claudioMDS/hyper-perform}}
\begin{figure}
			\centering
			\includegraphics[height=200px]{../Images/CodusMaximus_logo.jpg}
\end{figure}

	
\author{
	\textbf{Developers:} \\
	Claudio Da Silva	\emph{14205892}	\\
	Rohan Chhipa		\emph{14188377}	\\
	Avinash Singh		\emph{14043778}	\\
	Jason Gordon		\emph{14405025}	\\\\
}

\date{\textbf{Updated \today}}

\maketitle
\thispagestyle{empty}
\pagebreak

\tableofcontents
\pagebreak

\section{Introduction}
Many different tools are available for measuring the quality of products made, but very few tools exist which assess the quality of the people making said products. People play a huge role in a project, and trying to monitor each and every one becomes a tedious task which diverts man power away from other more critical tasks. Whether it be for an end of year evaluation, or attempting to assess the current status of a project, generating a report on a staff member can help keep up productivity, as well as get them any help they need in order to resume quality performance. By ensuring that there is constant quality performance from each individual on a project, one can increase project quality as well as reduce project risks such as loss of an important team member during a critical stage of a project's life-cycle. 

\section{Installation of software}

This installation guide assumes a Linux Server running Ubuntu 14 or higher, docker will later became available for easy install:

\subsection{WildFly}
The install WildFly 10 on your server: \\\\
Elevate to sudo permissions:

\begin{lstlisting}
sudo -s
\end{lstlisting}
Install Java JDK 8:
\begin{lstlisting}
aptitude update
aptitude install --with-recommends software-properties-common
add-apt-repository ppa:webupd8team/java
aptitude update
aptitude --with-recommends install oracle-java8-installer vim
\end{lstlisting}
Create a WildFly user:
\begin{lstlisting}
adduser --no-create-home --disabled-password --disabled-login wildfly
\end{lstlisting}
Download and extract the WildFly installation:
\begin{lstlisting}
cd /opt
wget --tries=0 --continue http://download.jboss.org/wildfly/10.0.0.Final/wildfly-10.0.0.Final.tar.gz
tar -xzvf wildfly-10.0.0.Final.tar.gz
ln -s wildfly-10.0.0.Final wildfly
chown -R wildfly wildfly
\end{lstlisting}
\pagebreak
Copy the initialization scripts to the required folders:
\begin{lstlisting}
cp /opt/wildfly/docs/contrib/scripts/init.d/wildfly-init-debian.sh /etc/init.d/wildfly
update-rc.d /etc/init.d/wildfly defaults
cp /opt/wildfly/docs/contrib/scripts/init.d/wildfly.conf /etc/default/wildfly
cd /etc/default
\end{lstlisting}
Now to edit the files:
\begin{lstlisting}
nano wildfly
\end{lstlisting}
Uncomment and/or Edit the following lines:
\begin{lstlisting}
JBOSS_HOME="/opt/wildfly"
JBOSS_USER=wildfly
JBOSS_MODE=standalone
JBOSS_CONFIG=standalone-full.xml
JBOSS_CONSOLE_LOG="/var/log/wildfly/console.log"
\end{lstlisting}
Change all 127.0.0.1 in standalone-full.xml to 0.0.0.0, then run the following commands:
\begin{lstlisting}
service wildfly start
cd /opt/wildfly/bin
./add-user.sh
\end{lstlisting}

\subsection{PostgreSQL}
The install PostgreSQL on your server: \\\\
Install via terminal:
\begin{lstlisting}
sudo apt-get update
sudo apt-get install postgresql postgresql-contrib
\end{lstlisting}
Then create the required database:
\begin{lstlisting}
sudo -i -u postgres
psql
createdb hyperperform
\end{lstlisting}
To configure PostgreSQL to connect remotely:
\begin{lstlisting}
sudo nano /etc/postgresql/9.3/main/postgresql.conf
\end{lstlisting}
Edit the following lines:
\begin{lstlisting}
listen_addresses = "*"
\end{lstlisting}



\subsection{ActiveMQ}
To setup ActiveMQ on your server: \\\\
\begin{itemize}
	\item Navigate to WildFly management console on localhost:9990
	\item Navigate to configurations tab and click on sub-systems
	\item Scroll down and search for Messaging-ActiveMQ and click on it
	\item Click on default, select queues/topics
	\item Click add and input the following information:
		\begin{itemize}
			\item Name*: hyperperform
			\item JNDI Names*: java:/jms/queue/hyperperform
		\end{itemize}
	\item Click save
\end{itemize}


\subsection{Angular2}
To install Angular2 on your server: \\\\
Start by setting up node version manager:
\begin{lstlisting}
sudo apt-get update
sudo apt-get install build-essential libssl-dev
sudo curl https://raw.githubusercontent.com/creationix/nvm/v0.32.1/install.sh | bash
source ~/.profile
\end{lstlisting}
Then setup node:
\begin{lstlisting}
nvm install 6.3.2
\end{lstlisting}
Then finish installing Angular2 with:
\begin{lstlisting}
npm install -g angular-cli
\end{lstlisting}

\pagebreak

\section{Installation of HyperPerform}

\subsection{Server}
To setup the back end, take the hyperperform.war and deploy it on the WildFly management console

\subsection{Dashboard}
To install the dashboard on your server:
\begin{lstlisting}
npm install hyperperform
\end{lstlisting}


\end{document}